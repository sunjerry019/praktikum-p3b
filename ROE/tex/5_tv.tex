\section{Teilversuch 5:Identifikation einer unbekannte Probe mittels Röntgen\-flu\-o\-res\-zenz}
	Im Versuch haben wir eine 1\texteuro-Münze verwendet und das Spektrum hat zu dem von \ce{Zn} gepasst. 

	Aus Wikipedia\footnote{\url{https://de.wikipedia.org/wiki/Eurom\%C3\%BCnzen}} besteht die Münze als folgende Materialien: 
	\begin{itemize}
		\item Ring: Nickel-Messing (75 \% Cu, 20 \% Zn, 5 \% Ni)
		\item Kern: Kupfernickel, Nickel, Kupfernickel geschichtet (Magnimat) 
	\end{itemize}
	Es scheint also, dass \ce{Zn} eigentlich kein größer Anteil ist, was ziemlich interessant ist. Vielleicht war die Münze von \ce{Zn} beschichtet und somit erhalten wir ein starke \ce{Zn} Signal.
	\vfill