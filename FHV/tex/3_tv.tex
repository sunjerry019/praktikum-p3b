\section{Teilversuch 3: Aufnahme des Spektrums der Neon-Glimm\-ent\-la\-dung}
	Diese Linien liegen alle innerhalb des sichtbaren Bereichs (ca. \num{380} bis \num{750}\si{\nano\meter}) \citep{starr_biology_2006} und sogar im orangen/roten Bereich. Es gibt auch sehr viele Spektralliniien. Genau weißt ich eigentlich nicht, warum das zu erwarten ist:
	\begin{itemize}
		\item Warum orangen/roten Bereich?
		\item Warum so viele Spektrallinien?
	\end{itemize}
	Ich kann mich aber mithilfe des Termschemas erwarten, dass die Spektrallinien genau die Energieunterschiede zwischen die Energieniveau sind. Bei einer Glimmentladung wird sehr viele Atomen angeregt. Sie zerfallen und produzieren wegen des genauen Unterschieds zwischen die Energieniveau auch entsprechend Photonen von bestimmten Wellenlängen. Da die Energieunterschiede im Bereich orange/rot liegen, erhalten wir auch durch die Glimmentladung diese Linien. 

	Ich würde mich jedoch freuen, wenn wir diese Aufgabe auch im Nachgespräch diskutieren können.


